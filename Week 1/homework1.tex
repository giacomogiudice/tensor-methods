\documentclass[a4paper,10pt,twoside]{article}
%%%%%%%%%%% Packages %%%%%%%%%%
\usepackage[margin=1in]{geometry}
\usepackage{amsmath, amssymb,mathtools}
\usepackage{fancyhdr}
\usepackage{sectsty}
\usepackage{enumitem}
\usepackage{bbm}
\usepackage{braket}


%%%%%%%%%%% Macros %%%%%%%%%%
\def \dd {{\rm d}}
\def \id {{\mathbbm{1}}}
\def \note#1 {\vspace{-1em}\paragraph{\bfseries #1}}
\DeclareMathOperator{\trace}{tr}

%%%%%%%%%%% Formatting %%%%%%%%%%
\pagestyle{fancy}
\renewcommand{\footrulewidth}{0.5pt}

\fancyhf{}
\lhead{20/04/2017}
\chead{Quantum Information Methods in Many-Body Physics}
\rhead{PH2269}
\lfoot{Giacomo Giudice~~~~giacomo.giudice@mpq.mpg.de}
\rfoot{Page \thepage}

\allsectionsfont{\normalfont\sffamily}

%%%%%%%%%%% Here Begins Document %%%%%%%%%%
\begin{document}
\title{\vspace{-1cm}\sffamily Homework 1\vspace{-1cm}}
\author{}
\date{}
\maketitle
\thispagestyle{fancy}

\begin{section}{Every Mixed State is a Reduced State}
A generic density operator $\rho$ can be written as $\rho = \sum_n \lambda_n \ket{\varphi_n}\bra{\varphi_n}$, with $\{ \varphi_n \}$ a complete set of basis states satisfying $\braket{\varphi_n | \varphi_m} = \delta_{n,m}$.
Explain why this is possible and why all the eigenvalues $\lambda_n$ are real and $\lambda_n \geq 0$.

We will now show that it is always possible to construct a purification of some system $A$ by introducing an additional copy of the system called $B$.
Starting from the pure state
\[ 
  \ket{\Psi}_{A B} = \sum_n \sqrt{\lambda_n} \ket{\varphi_n}_A \ket{\varphi_n}_B
\]
show that we obtain the reduced state $\rho$ for $A$ by tracing out $B$.
\note{Bonus} This was obtained with an ancilla of the same dimension as the system. Can we obtain a purification with an ancilla of a smaller dimension? What constraints would that introduce on $\rho$?
\end{section}

\begin{section}{Mutual Information Inequalities}
Let $A_1$, $B$ and $C$ be three quantum systems. 
Prove the following inequalities:
\begin{enumerate}[label=(\alph*)]
\item $I(A,B:C) \geq I(B:C)$,
\item $I(A,B:C) \leq I(B:C) + 2S(A)$.
\end{enumerate}
\end{section}

\begin{section}{A Two-Qubit State}
Compute the entanglement of the two-qubit state
\[
  \ket{\Psi} = \frac{-(1 - 2\sqrt{2})\ket{00}  - (2 + \sqrt{2})\ket{10} + (1+2\sqrt{2})\ket{01} + (2 - \sqrt{2})\ket{11}}{\sqrt{30}} .
\]
\end{section}
\begin{section}{Something That Will Come Back Later On}
For spin-1/2 systems, we denote the Pauli matrices as
\[
  \sigma^x=
  \begin{pmatrix}
    0 & 1\\
    1 & 0
  \end{pmatrix},
  \quad
  \sigma^y=
  \begin{pmatrix}
    0 & -i\\
    i & 0
  \end{pmatrix},
  \quad
  \sigma^z=
  \begin{pmatrix}
    1 & 0\\
    0 & -1
  \end{pmatrix}.
\]
There, I wrote them down once, and I won't bother doing it in the next assignments.
Additionally, we define the raising and lowering operators as $\sigma^\pm = (\sigma^x \pm i\sigma^y)/2$.

Let $\{\ket{+}, \ket{0}, \ket{-}\}$ be the standard basis for a spin-1 system. 
We will now construct a two-particle state in a slightly unusual way as 
\[
  \ket{\Psi} = \sum_{\mathclap{ i,j = +,0,-}} c_{ij} \ket{i j}, \quad{\rm where} \; c_{i j} = \frac{1}{\sqrt{3}}\trace{\left(A^i A^j\right)}
\]
and the matrices $\{ A^i \}$ are defined as
\[
  A^+ = \sigma^+, \quad A^0 = -\frac{1}{\sqrt{2}} \sigma^z, \quad A^- = -\sigma^- \,.
\]
Compute the entanglement between the two particles.
%\note{Note} You can write a computer programs to help you with the linear algebra.
\end{section}


\end{document}
%%%%%%%%%%% Here Ends Document %%%%%%%%%%
