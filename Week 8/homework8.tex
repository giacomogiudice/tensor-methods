\documentclass[a4paper,10pt,twoside]{article}
%%%%%%%%%%% Packages %%%%%%%%%%
\usepackage[margin=1in]{geometry}
\usepackage{amsmath, amssymb,mathtools}
\usepackage{fancyhdr}
\usepackage{sectsty}
\usepackage{graphicx,wrapfig}
\usepackage{enumitem}
\usepackage{float}
\usepackage{braket}
\usepackage{bbm}
\usepackage{tikz,calc}

%%%%%%%%%%% Macros %%%%%%%%%%
\def \note#1 {\vspace{-1em}\paragraph{\bfseries #1}}
\def \dd {{\rm d}}
\def \id {{\mathbbm{1}}}
\def \order {\mathcal{O}}
\def\bquad{\mkern-18mu}
\DeclareMathOperator{\trace}{tr}
\DeclareMathOperator{\spanset}{span}

%%%%%%%%%%% Tikz Definitions %%%%%%%%%%
\usetikzlibrary{shapes, arrows,positioning,fit}
\tikzstyle{plain} = [draw,thick,circle,inner sep=0,minimum size=0.5cm,font=\footnotesize]
\tikzstyle{mps} = [draw,thick,rectangle,rounded corners=.1cm,inner sep=0,minimum size=0.5cm]
\tikzstyle{mpo} = [draw,thick,circle,inner sep=0,minimum size=0.5cm]
\tikzstyle{index} = [-,thick,font=\footnotesize]
\tikzstyle{virtual} = [-,thick,dotted,font=\footnotesize]
\tikzstyle{site} = [draw,solid,circle,minimum size=2pt,inner sep=0pt,outer sep=0pt,fill=black]

\def \tu {0.25cm}

%%%%%%%%%%% Formatting %%%%%%%%%%
\pagestyle{fancy}
\renewcommand{\footrulewidth}{0.5pt}

\fancyhf{}
\lhead{29/07/2017}
\chead{Quantum Information Methods in Many-Body Physics}
\rhead{PH2269}
\lfoot{Giacomo Giudice~~~~giacomo.giudice@mpq.mpg.de}
\rfoot{Page \thepage}

\allsectionsfont{\normalfont\sffamily}

%%%%%%%%%%% Here Begins Document %%%%%%%%%%
\begin{document}
\title{\vspace{-1cm}\sffamily Homework 8\vspace{-1cm}}
\author{}
\date{}
\maketitle
\thispagestyle{fancy}

\begin{section}{Trotter--Suzuki Decompositions}
For two operators $A$ and $B$, the Baker--Campbell--Hausdorff formula gives a form for the composition of their two exponentials  $e^{A}e^{B} = e^{C}$, in terms of an infinite series of commutators, where 
\[
  C = A + B + \frac{1}{2}[A,B] + \frac{1}{12} \Big( [A,[A,B]] + [B,[B,A]] \Big) - \frac{1}{24} [B,[A,[A,B]]] \dots
\]
This series seems quite complicated, but it is actually quite amenable to a perturbative expansion since it can be viewed as a series of increasing orders of some small parameter, which we call $\tau$.
Consider a time-independent Hamiltonian $H$ that can be decomposed as the sum of two other operators $H_1$ and $H_2$, i.e. $H = H_1 + H_2$.
What is the first order approximation of the evolution operator $U(\tau) = e^{ -i\tau(H_1 + H_2)}$ in terms of $U_{1,2}(\tau) = \exp(-i \tau H_{1,2})$?

One can improve this integrations scheme by choosing a combination of time steps in such a way to cancel out the next leading order.
Verify the second-order Trotter--Suzuki decomposition
\[
  U(\tau) = U_1(\tau/2) \, U_2(\tau) \,U_1(\tau/2) + \order(\tau^3) .
\]
Higher order expansions are possible but are a bit tedious to calculate\footnote{In general, one can pose
\[
  U(\tau) = U_1(c_1 \tau) \, U_2(d_1 \tau) \, U_1(c_2 \tau) \, U_2(d_2 \tau) \dots \, U_1(c_k \tau) \, U_2(d_k\tau) + \order(\tau^{k+1}) .
\]
and insert it in the BCH formula with the normalization constraints $\sum_i c_i = \sum_i d_i = 1$.
The complexity increases very quickly: for example the $\order(\tau^5)$ expansion has 8 equations with $\sim 100$ terms in each. The solution, known as the Forest--Ruth decomposition, circulated unpublished in the particle accelerator community for several years.}.
Are these higher-order expansions useful for for time-evolution algorithms in tensor networks?
Consider the drawbacks, taking in account the fact that at some point one has to truncate the bond dimension.
\end{section}

\begin{section}{Why We Care So Much About SVD}
One of the successes of tensor networks in quantum information is attributed to their capability of representing states which belong to a small but physically relevant corner of an otherwise prohibitively large Hilbert space, for example where entanglement is low.
The backbone concepts is that of the \emph{low-rank approximation}.
Suppose you have some matrix $M$, and you want to compute the closest possible approximation $\tilde{M}^{(k)}$ of rank $k$, which minimizes $\|M -\tilde{M}^{(k)}\|_F$ . 
The Eckart--Young theorem\footnote{This result was previously found by Schmidt. It was subsequently generalized by Mirsky to show optimality for unitarily-invariant norms.} states that the optimal choice, under the Frobenius norm, is given by the trimmed down singular value decomposition 
\[
  \tilde{M}^{(k)} = U\tilde{S}^{(k)} V^\dag,
\]
such that $M = USV^\dag$ and $\tilde{S}^{(k)}$ has all but the largest $k$ singular values set to zero.

Now consider an MPS $\ket{\Psi}$, with open boundary conditions.
Compress a single bond, to find the best approximation $\ket{\tilde{\Psi}}$, and show that 
\[
  \braket{\Psi|\tilde{\Psi}} = \trace{\left((\tilde{S}^{(k)})^2\right)}.
\]
What does this imply for the distance $\|\ket{\Psi} - \ket{\tilde{\Psi}} \|^2$?
\end{section}

\begin{section}{AKLT Reloaded*}
In this exercise, we will be numerically studying the properties of the AKLT state $\ket{\Psi}$ as an MPS (see Homework 3).

\begin{enumerate}[label=(\alph*)]
  \item Since the state is translationally invariant, you should just store a single tensor $A$. To make sure you defined it correctly, check the gauge condition.
  \item Construct the transfer matrix $\mathbb{E}$ and its eigenvalues. You should obtain $(1,-1/3,-1/3,-1/3)$.
  \item Compute the dominant eigenvector $\ket{\rho}$ of the transfer matrix, and normalize it such that $\trace{\rho} = 1$.
  \item Compute the behavior of the following correlators
  \begin{align*}
    \chi^{zz}(r) &= \braket{\Psi | S^z_i S^z_j | \Psi} ,\\
    \chi^{+-}(r) &= \braket{\Psi | S^+_i S^-_j | \Psi} ,\\ 
    \chi^{\rm string}(r) &= \braket{\Psi | S^z_i e^{i \pi \sum_{i<\ell<j} S^z_\ell} S^z_j | \Psi} ,
  \end{align*}
  for increasing values of $r=|i-j|$ and plot their decay as a function of $r$. 
  You should do this by defining the transfer matrix for local operators $\mathbb{E}_O = \sum_{i,j} O_{ij}  \bar{A}^{i} \otimes A^{j}$.
\end{enumerate}
You should obtain the following behaviors: $\chi^{zz}(r) \sim e^{-r/\xi}$, $\chi^{+-}(r) \sim e^{-r/\xi}$, $\xi = 1/\log 3$, while $\chi^{\rm string}(r) = -4/9$.

These results (which, by the way, are computable analytically), indicate that correlation functions of any local operator decay exponentially (can you guess the decay rate directly from the transfer matrix?). 
However, there is some kind of hidden order measurable with the string correlator (commonly known as string-order parameter). 
Can you guess the nature of this hidden order?

\end{section}
\end{document}
%%%%%%%%%%% Here Ends Document %%%%%%%%%%
