\documentclass[a4paper,10pt,twoside]{article}
%%%%%%%%%%% Packages %%%%%%%%%%
\usepackage[margin=1in]{geometry}
\usepackage{amsmath, amssymb,mathtools}
\usepackage{fancyhdr}
\usepackage{sectsty}
\usepackage{graphicx,wrapfig}
\usepackage{enumitem}
\usepackage{float}
\usepackage{braket}
\usepackage{bbm}
\usepackage{tikz,calc}
\usepackage{amsthm}


%%%%%%%%%%% Macros %%%%%%%%%%
\def \note#1 {\vspace{-1em}\paragraph{\bfseries #1}}
\def \dd {{\rm d}}
\def \id {{\mathbbm{1}}}
\def \order {\mathcal{O}}
\def\bquad{\mkern-18mu}
\DeclareMathOperator{\trace}{tr}
\DeclareMathOperator{\spanset}{span}

%%%%%%%%%%% Tikz Definitions %%%%%%%%%%
\usetikzlibrary{shapes, arrows,positioning,fit}
\tikzstyle{plain} = [draw,thick,circle,inner sep=0,minimum size=0.5cm,font=\footnotesize]
\tikzstyle{mps} = [draw,thick,rectangle,rounded corners=.1cm,inner sep=0,minimum size=0.5cm]
\tikzstyle{mpo} = [draw,thick,circle,inner sep=0,minimum size=0.5cm]
\tikzstyle{index} = [-,thick,font=\footnotesize]
\tikzstyle{virtual} = [-,thick,dotted,font=\footnotesize]

\def \tu {0.25cm}

%%%%%%%%%%% Formatting %%%%%%%%%%
\pagestyle{fancy}
\renewcommand{\footrulewidth}{0.5pt}

\fancyhf{}
\lhead{11/05/2017}
\chead{Quantum Information Methods in Many-Body Physics}
\rhead{PH2269}
\lfoot{Giacomo Giudice~~~~giacomo.giudice@mpq.mpg.de}
\rfoot{Page \thepage}

\allsectionsfont{\normalfont\sffamily}

\newtheoremstyle{modern}{3pt}{3pt}{\itshape}{}{\sffamily\bfseries}{}{.5em}{}
\theoremstyle{modern}
\newtheorem{lemma}{Lemma}[section]
\newtheorem{theorem}[lemma]{Theorem}

%%%%%%%%%%% Here Begins Document %%%%%%%%%%
\begin{document}
\title{\vspace{-1cm}\sffamily Homework 4\vspace{-1cm}}
\author{}
\date{}
\maketitle
\thispagestyle{fancy}

\begin{section}{The Cluster State}
The group $\mathrm{Z}_2 \times \mathrm{Z}_2$ has the presentation $\mathrm{Z}_2 \times \mathrm{Z}_2 = \Braket{ x,z | x^2 = z^2 = \mathbf{1}, xz = zx}$, where $\mathbf{1}$ is the identity element.
\begin{enumerate}[label=(\alph*)]
\item Show that the Pauli matrices $\{X,Y,Z\}$ form a \emph{projective representation} of this group, i.e. $v_g v_h = e^{i \omega(g,h)} v_{gh}$.
\item What are the possible values of $e^{i \omega(g,h)}$?
\item Can the Pauli representation form a \emph{linear representation} of this group? 
That is, can we rephase $v_g \to e^{i\phi_g} v_g$ to make the commutator zero?
\end{enumerate}
\note{Hint} Define $v_x = X$, $v_z = Z$, $v_{xz} = Y$.

The cluster state Hamiltonian is
\[
  H_c = - \sum_{i=1}^N Z_{i-1} X_i Z_{i+1} .
\]
The ground state $\ket{\psi_c}$ is an MPS of bond dimension $D=2$. 
In the computational basis,
\[
  A^0 = \ket{0}\bra{+}, \quad A^1 = \ket{1}\bra{-}, \quad \mbox{where}\ \ket{\pm} = \frac{\ket{0} \pm \ket{1}}{\sqrt{2}}.
\]
There are two operations that leave the state invariant, namely
\[
  X_{\rm odd} = \bigotimes_{i~{\rm odd}} X_i, \quad X_{\rm even} = \bigotimes_{i~{\rm even}} X_i.
\]
Check if these operators commute with the Hamiltonian.
What is the \emph{symmetry group} of the cluster state?
Show that it is $\mathrm{Z}_2 \times \mathrm{Z}_2$.
Now prove the following properties:
\[
  {
  \tikz[baseline=0,node distance=0.5*\tu]{
    \node[mps,] (a) {$A$};
    \node[plain,above=of a] (u) {$X$};
    \draw[index] (a.west) -- +(-\tu,0);
    \draw[index] (a.east) -- +(\tu,0);
    \draw[index] (u.north) -- +(0,\tu);
    \draw[index] (a.north) -- (u.south);
    }
  }
  = 
  {
  \tikz[baseline=0,node distance=\tu]{
    \node[mps,] (a) {$A$};
    \node[plain,left=of a] (v) {$X$};
    \node[plain,right=of a] (vdag) {$Z$};
    \draw[index] (v.west) -- +(-\tu,0);
    \draw[index] (vdag.east) -- +(\tu,0);
    \draw[index] (v.east) -- +(\tu,0);
    \draw[index] (a.east) -- +(\tu,0);
    \draw[index] (a.north) -- +(0,\tu);
    \draw[index] (v.east) -- (a.west);
    \draw[index] (vdag.west) -- (a.east);
    }
  }\ ,
  \qquad
  {
  \tikz[baseline=0,node distance=\tu]{
    \node[mps,] (a) {$A$};
    \node[plain,left=of a] (v) {$Z$};
    \node[plain,right=of a] (vdag) {$X$};
    \draw[index] (v.west) -- +(-\tu,0);
    \draw[index] (vdag.east) -- +(\tu,0);
    \draw[index] (v.east) -- +(\tu,0);
    \draw[index] (a.east) -- +(\tu,0);
    \draw[index] (a.north) -- +(0,\tu);
    \draw[index] (v.east) -- (a.west);
    \draw[index] (vdag.west) -- (a.east);
    }
  }
  =
  {
  \tikz[baseline=0,node distance=\tu]{
    \node[mps,] (a) {$A$};
    \draw[index] (a.west) -- +(-\tu,0);
    \draw[index] (a.east) -- +(\tu,0);
    \draw[index] (a.north) -- +(0,\tu);
    }
  }\ .
\]
By blocking two sites, show that 
\[
  {
  \tikz[baseline=0,node distance=\tu]{
    \node[mps] at (\tu,0) (m1) {$A$};
    \node[mps,right=\tu of m1] (m2) {$A$};
    \node[plain,above=0.5*\tu of m1] (u) {$X$};
    \draw[index] (m1.west) -- +(-\tu,0);
    \draw[index] (m2.east) -- +(\tu,0);
    \draw[index] (m2.north) -- +(0,\tu);
    \draw[index] (m1.north) -- (u.south);
    \draw[index] (m1.east) -- (m2.west);
    \draw[index] (u.north) -- +(0,\tu);
    }
  }
  = 
  {
  \tikz[baseline=0,node distance=\tu]{
    \node[mps] at (\tu,0) (m1) {$A$};
    \node[mps,right=\tu of m1] (m2) {$A$};
    \node[plain,left=of m1] (v) {$X$};
    \node[plain,right=of m2] (vdag) {$X$};
    \draw[index] (m1.east) -- (m2.west);
    \draw[index] (m1.north) -- +(0,\tu);
    \draw[index] (m2.north) -- +(0,\tu);
    \draw[index] (v.west) -- +(-\tu,0);
    \draw[index] (vdag.east) -- +(\tu,0);
    \draw[index] (m1.west) -- (v.east);
    \draw[index] (m2.east) -- (vdag.west);
    }
  }\ ,
  \qquad
  {
  \tikz[baseline=0,node distance=\tu]{
    \node[mps] at (\tu,0) (m1) {$A$};
    \node[mps,right=of m1] (m2) {$A$};
    \node[plain,above=0.5*\tu of m2] (u) {$X$};
    \draw[index] (m1.west) -- +(-\tu,0);
    \draw[index] (m2.east) -- +(\tu,0);
    \draw[index] (m1.north) -- +(0,\tu);
    \draw[index] (m2.north) -- (u.south);
    \draw[index] (m1.east) -- (m2.west);
    \draw[index] (u.north) -- +(0,\tu);
    }
  }
  = 
  {
  \tikz[baseline=0,node distance=\tu]{
    \node[mps] at (\tu,0) (m1) {$A$};
    \node[mps,right=of m1] (m2) {$A$};
    \node[plain,left=of m1] (v) {$Z$};
    \node[plain,right=of m2] (vdag) {$Z$};
    \draw[index] (m1.east) -- (m2.west);
    \draw[index] (m1.north) -- +(0,\tu);
    \draw[index] (m2.north) -- +(0,\tu);
    \draw[index] (v.west) -- +(-\tu,0);
    \draw[index] (vdag.east) -- +(\tu,0);
    \draw[index] (m1.west) -- (v.east);
    \draw[index] (m2.east) -- (vdag.west);
    }
  }\ ,
\]
and explicitly write the different local representations $u_g$ and their associated matrices $V_g$.
Graphically argue that cluster state is invariant under the operations in the symmetry group.
\end{section}

\begin{section}{Gauges and Symmetries}
Remember that the gauge freedom for MPS is $A \mapsto G A G^{-1}$.
Hence, we could imagine a unitary group acting as
\[
  {
  \tikz[baseline=0*\tu,node distance=0.5*\tu]{
    \node[mps,] (a) {$A$};
    \node[plain,above=of a] (u) {$u_g$};
    \draw[index] (a.west) -- +(-\tu,0);
    \draw[index] (a.east) -- +(\tu,0);
    \draw[index] (u.north) -- +(0,\tu);
    \draw[index] (a.north) -- (u.south);
    }
  }
  = 
  {
  \tikz[baseline=0*\tu,node distance=\tu]{
    \node[mps,] (a) {$A$};
    \node[plain,left=of a] (v) {$Y_g$};
    \node[plain,right=of a] (vdag) {$Y^{\scalebox{0.6}{-1}}_g$};
    \draw[index] (v.west) -- +(-\tu,0);
    \draw[index] (vdag.east) -- +(\tu,0);
    \draw[index] (v.east) -- +(\tu,0);
    \draw[index] (a.east) -- +(\tu,0);
    \draw[index] (a.north) -- +(0,\tu);
    \draw[index] (v.east) -- (a.west);
    \draw[index] (vdag.west) -- (a.east);
    }
  }
\]
However, by choosing a canonical form, this imposes the constraint that $Y_g$ be unitary.
Show this, by taking the equation defining a canonical form, and assuming injective tensors.

\note{Hint} For injective tensor the left and right fixed points are unique.
\end{section}

\begin{section}{The AKLT State, Because We Never Saw It Before}
Construct the parent Hamiltonian of the AKLT state as the projector on the spin-2 subspace of two spin-1 particles (labeled 1 \& 2). 
You should obtain 
\[
  h_{1,2} = \frac{1}{2} \vec{S}_1 \cdot \vec{S}_2 + \frac{1}{6}  (\vec{S}_1 \cdot \vec{S}_2)^2 + \frac{1}{3}
\]
\note{Hint} You may do this by brute force, or by being a little bit smart.
Start by expanding $(\vec{S}_1 + \vec{S}_2)^2$.
What are its possible eigenvalues?
Now, either construct a projector onto $S=2$ as a product of terms involving $(\vec{S}_1 + \vec{S}_2)^2$, or build a polynomial in terms of $X = \vec{S}_1 \cdot \vec{S}_2$ that satisfies the constraints above. 
\end{section}
\end{document}
%%%%%%%%%%% Here Ends Document %%%%%%%%%%
