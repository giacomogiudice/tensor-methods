\documentclass[a4paper,10pt,twoside]{article}
%%%%%%%%%%% Packages %%%%%%%%%%
\usepackage[margin=1in]{geometry}
\usepackage{amsmath, amssymb,mathtools}
\usepackage{fancyhdr}
\usepackage{sectsty}
\usepackage{graphicx,wrapfig}
\usepackage{enumitem}
\usepackage{float}
\usepackage{braket}
\usepackage{bbm}
\usepackage{tikz,calc}
\usepackage{amsthm}


%%%%%%%%%%% Macros %%%%%%%%%%
\def \note#1 {\vspace{-1em}\paragraph{\bfseries #1}}
\def \dd {{\rm d}}
\def \id {{\mathbbm{1}}}
\def \order {\mathcal{O}}
\def\bquad{\mkern-18mu}
\DeclareMathOperator{\trace}{tr}
\DeclareMathOperator{\spanset}{span}

%%%%%%%%%%% Tikz Definitions %%%%%%%%%%
\usetikzlibrary{shapes, arrows,positioning,fit}
\tikzstyle{plain} = [draw,thick,circle,inner sep=0,minimum size=0.5cm,font=\footnotesize]
\tikzstyle{mps} = [draw,thick,rectangle,rounded corners=.1cm,inner sep=0,minimum size=0.5cm]
\tikzstyle{mpo} = [draw,thick,circle,inner sep=0,minimum size=0.5cm]
\tikzstyle{index} = [-,thick,font=\footnotesize]
\tikzstyle{virtual} = [-,thick,dotted,font=\footnotesize]

\def \tu {0.25cm}

%%%%%%%%%%% Formatting %%%%%%%%%%
\pagestyle{fancy}
\renewcommand{\footrulewidth}{0.5pt}

\fancyhf{}
\lhead{11/05/2017}
\chead{Quantum Information Methods in Many-Body Physics}
\rhead{PH2269}
\lfoot{Giacomo Giudice~~~~giacomo.giudice@mpq.mpg.de}
\rfoot{Page \thepage}

\allsectionsfont{\normalfont\sffamily}

\newtheoremstyle{modern}{3pt}{3pt}{\itshape}{}{\sffamily\bfseries}{}{.5em}{}
\theoremstyle{modern}
\newtheorem{lemma}{Lemma}[section]
\newtheorem{theorem}[lemma]{Theorem}

%%%%%%%%%%% Here Begins Document %%%%%%%%%%
\begin{document}
\title{\vspace{-1cm}\sffamily Solutions to Homework 4\vspace{-1cm}}
\author{}
\date{}
\maketitle
\thispagestyle{fancy}

\begin{section}{}
\begin{enumerate}[label=(\alph*)]
\item By defining $v_x = X$, $v_z = Z$, $v_{xz} = Y$,
we obtain the following \emph{factor system} $e^{i \omega(g,h)}$, represented as a matrix,
\[
  \left\{e^{i \omega(g,h)}\right\}_{g,h} = 
  \begin{pmatrix}
    1 & 1 & 1 & 1 \\
    1 & 1 & -i & i \\
    1 & i & 1 & -i \\
    1 & i & -i & 1 \\
  \end{pmatrix}, \quad g,h = \mathbf{1},x,z,{xz} .
\]
\item See above.
\item By changing the phase $v_g \to e^{i\phi_g} v_g$ the commutator becomes $e^{i (\phi_x + \phi_z)}(v_x v_z - v_z v_x)$ which is non-zero for every value of $\phi_x + \phi_z$.
Since no rephasing can bring the commutator to zero, we deduce that the Pauli matrices cannot form a linear representation of this group.
\end{enumerate}
The symmetry of the cluster state is $G = \{ \id, X_{\rm odd}, X_{\rm even}, X_{\rm all}\}$, where $X_{\rm all} = X_{\rm even} \!\circ X_{\rm odd}$ (the group is abelian and these two terms commute). Notice that this group is $\mathrm{Z}_2 \times \mathrm{Z}_2$, since $X_{\rm even}^2 = X_{\rm odd}^2 = \id$, $[X_{\rm even} , X_{\rm odd}] = 0$.

The properties of the cluster state are easily checked by using the properties of the Pauli matrices.
By representing each entry $A^i$ in a vector,
\[
  {
  \tikz[baseline=0,node distance=\tu]{
    \node[mps,] (a) {$A$};
    \draw[index] (a.west) -- +(-\tu,0);
    \draw[index] (a.east) -- +(\tu,0);
    \draw[index] (a.north) -- +(0,\tu);
    }
  }
  =
  \begin{pmatrix}
    A^0 \\
    A^1
  \end{pmatrix}
\]
we obtain
\begin{align*}
  {
  \tikz[baseline=0,node distance=\tu]{
    \node[mps,] (a) {$A$};
    \node[plain,left=of a] (v) {$X$};
    \node[plain,right=of a] (vdag) {$Z$};
    \draw[index] (v.west) -- +(-\tu,0);
    \draw[index] (vdag.east) -- +(\tu,0);
    \draw[index] (v.east) -- +(\tu,0);
    \draw[index] (a.east) -- +(\tu,0);
    \draw[index] (a.north) -- +(0,\tu);
    \draw[index] (v.east) -- (a.west);
    \draw[index] (vdag.west) -- (a.east);
    }
  }
  &= 
  \begin{pmatrix}
    X \ket{0}\bra{+} Z \\
    X \ket{1}\bra{-} Z
  \end{pmatrix}
  =
  \begin{pmatrix}
    \ket{1}\bra{-} \\
    \ket{0}\bra{+}
  \end{pmatrix}
  =  
  {
  \tikz[baseline=0,node distance=0.5*\tu]{
    \node[mps,] (a) {$A$};
    \node[plain,above=of a] (u) {$X$};
    \draw[index] (a.west) -- +(-\tu,0);
    \draw[index] (a.east) -- +(\tu,0);
    \draw[index] (u.north) -- +(0,\tu);
    \draw[index] (a.north) -- (u.south);
    }
  }\ ,\\
  {
  \tikz[baseline=0,node distance=\tu]{
    \node[mps,] (a) {$A$};
    \node[plain,left=of a] (v) {$Z$};
    \node[plain,right=of a] (vdag) {$X$};
    \draw[index] (v.west) -- +(-\tu,0);
    \draw[index] (vdag.east) -- +(\tu,0);
    \draw[index] (v.east) -- +(\tu,0);
    \draw[index] (a.east) -- +(\tu,0);
    \draw[index] (a.north) -- +(0,\tu);
    \draw[index] (v.east) -- (a.west);
    \draw[index] (vdag.west) -- (a.east);
    }
  }
  &=
  \begin{pmatrix}
    Z \ket{0}\bra{+} X \\
    Z \ket{1}\bra{-} X
  \end{pmatrix}
  = 
  \begin{pmatrix}
    \ket{0}\bra{+} \\
    \ket{1}\bra{-}
  \end{pmatrix}
  =
  {
  \tikz[baseline=0,node distance=\tu]{
    \node[mps,] (a) {$A$};
    \draw[index] (a.west) -- +(-\tu,0);
    \draw[index] (a.east) -- +(\tu,0);
    \draw[index] (a.north) -- +(0,\tu);
    }
  }\ .
\end{align*}
We can use these properties to define the action of the symmetries on the pair of sites:
\begin{align*}
  (X_{i,i^\prime} \otimes \id_{j,j^\prime}) (A^{i^\prime} A^{j^\prime}) &= X A^i Z A^j (X X) = X (A^i A^j) X, \\
  (\id_{i,i^\prime} \otimes X_{j,j^\prime}) (A^{i^\prime} A^{j^\prime}) &=  (Z Z) A^i X A^j Z = Z (A^i A^j) Z, \\
  (X_{i,i^\prime} \otimes X_{j,j^\prime}) (A^{i^\prime} A^{j^\prime}) &= X A^i Z X A^j Z =  X Z A^i (X Z)^2 A^j X Z = -iY (A^i A^j) iY.
\end{align*}
Notice that the last one can be deduced immediately, since it the composition of the two previous symmetries, therefore $Z X = -i Y$.
The different local representations $u_g$ and their actions on the virtual level can be summarized in the following table:
\begin{center}
\begin{tabular}{c|cccc}
 $u_g$ & $\id \otimes \id$ & $X \otimes \id$ & $\id \otimes X$ & $X \otimes X$ \\
 \hline
 $V_g$ & $\id$ & $X$ & $Z$ & $-iY$ 
\end{tabular}
\end{center}
These local operations are the building blocks for the global symmetries in $G$, as one can easily verify graphically.
\end{section}

\begin{section}{}
We can think of the action of $u_g$ as ``passing $Y_g$ through $A$''
\[
  {
  \tikz[baseline=0*\tu,node distance=\tu]{
    \node[mps,] (a) {$A$};
    \node[plain,above=0.5*\tu of a] (u) {$u_g$};
    \node[plain,right=of a] (v) {$Y_g$};
    \draw[index] (a.west) -- +(-\tu,0);
    \draw[index] (u.north) -- +(0,\tu);
    \draw[index] (v.east) -- +(\tu,0);
    \draw[index] (a.north) -- (u.south);
    \draw[index] (v.west) -- (a.east);
    }
  }
  = 
  {
  \tikz[baseline=0*\tu,node distance=\tu]{
    \node[mps,] (a) {$A$};
    \node[plain,left=of a] (v) {$Y_g$};
    \draw[index] (v.west) -- +(-\tu,0);
    \draw[index] (v.east) -- +(\tu,0);
    \draw[index] (a.east) -- +(\tu,0);
    \draw[index] (a.north) -- +(0,\tu);
    \draw[index] (v.east) -- (a.west);
    }
  }
\]
In the right canonical form, the \emph{only} fixed point is
\[
  {
  \tikz[baseline=-2*\tu,node distance=\tu]{
      \node[mps] (mconj) {$\bar{A}$};
      \node[mps,below=of mconj] (m) {$A$};
      \draw[index] (m.east) to[in=0,out=0] (mconj.east);
      \draw[index] (m.north) -- (mconj.south);
      \draw[index] (mconj.west) -- +(-\tu,0);
      \draw[index] (m.west) -- +(-\tu,0);
    }
  }
  =
  {
  \tikz[baseline=-2*\tu,node distance=\tu]{
      \node[] (gconj) {};
      \node[below=2*\tu of gconj] (g) {};
      \draw[index] (g.east) to[in=0,out=0] (gconj.east);
    }
  }\ .
\]
By passing $Y_g$ through $A$ we obtain
\[
  {
  \tikz[baseline=-0.4*\tu,node distance=\tu]{
      \node[plain] (rho) {$Y_g Y_g^\dag$};
      \draw[index] (rho.north) to[in=0,out=90] +(-0.5*\tu,0.5*\tu);
      \draw[index] (rho.south) to[in=0,out=-90] +(-0.5*\tu,-0.5*\tu);
    }
  }
  =
  {
  \tikz[baseline=-2*\tu,node distance=\tu]{
      \node[plain] (gconj) {$\bar{Y}_g$};
      \node[plain,below= of gconj] (g) {$Y_g$};
      \draw[index] (g.east) to[in=0,out=0] (gconj.east);
      \draw[index] (gconj.west) -- +(-\tu,0);
      \draw[index] (g.west) -- +(-\tu,0);
    }
  }
  =
  {
  \tikz[baseline=-2*\tu,node distance=\tu]{
      \node[mps] (mconj) {$\bar{A}$};
      \node[mps,below=of mconj] (m) {$A$};
      \node[plain,left=of mconj] (gconj) {$\bar{Y}_g$};
      \node[plain,left=of m] (g) {$Y_g$};
      \draw[index] (m.east) to[in=0,out=0] (mconj.east);
      \draw[index] (m.north) -- (mconj.south);
      \draw[index] (gconj.east) -- (mconj.west);
      \draw[index] (g.east) -- (m.west);
      \draw[index] (gconj.west) -- +(-\tu,0);
      \draw[index] (g.west) -- +(-\tu,0);
    }
  }
  =
  {
  \tikz[baseline=-2*\tu,node distance=\tu]{
      \node[plain] (mconj) {$\bar{Y}_g$};
      \node[plain,below=of mconj] (m) {$Y_g$};
      \node[mps,left=of mconj] (gconj) {$\bar{A}$};
      \node[mps,left=of m] (g) {$A$};
      \draw[index] (m.east) to[in=0,out=0] (mconj.east);
      \draw[index] (g.north) -- (gconj.south);
      \draw[index] (gconj.east) -- (mconj.west);
      \draw[index] (g.east) -- (m.west);
      \draw[index] (gconj.west) -- +(-\tu,0);
      \draw[index] (g.west) -- +(-\tu,0);
    }
  }
  =
  {
  \tikz[baseline=-2*\tu,node distance=\tu]{
      \node[mps] (mconj) {$\bar{A}$};
      \node[mps,below=of mconj] (m) {$A$};
      \node[fit=(mconj)(m)] (f) {};
      \node[plain,right=0 of f] (rho) {$Y_g Y_g^\dag$};
      \draw[index] (rho.north west) to[in=0,out=145] (mconj.east);
      \draw[index] (rho.south west) to[in=0,out=-145] (m.east);
      \draw[index] (m.north) -- (mconj.south);
      \draw[index] (mconj.west) -- +(-\tu,0);
      \draw[index] (m.west) -- +(-\tu,0);
    }
  }\ ,
\]
where the two $u_g$ cancel out since they are unitary.
Now, this equation means that $Y_g Y_G^\dag$ is a fixed point of the map. 
By the unicity of the fixed point, $Y_g Y_G^\dag = \alpha \id$, with $\alpha$ a positive, real constant (since $Y_g Y_G^\dag \succeq 0$) which can be absorbed in $Y_g$.
\end{section}

\begin{section}{}

Since $(\vec{S}_1 + \vec{S}_2)^2$ is the total spin has the eigenvalue equation
\[
  (\vec{S}_1 + \vec{S}_2)^2\ket{s,m_s} = \left(|S_1|^2 + 2 \vec{S}_1 \cdot \vec{S}_2 + |S_2|^2\right)\ket{s,m_s} = s(s+1)\ket{s,m_s} , \quad s = 0,1,2.
\]
Since $|S_1|^2 = |S_2|^2 = 2$, the scalar product can have the eigenvalues
\[
  \vec{S}_1 \cdot \vec{S}_2 = 
  \begin{cases}
    -2 & s=0\\
    -1 & s=1\\
    2 & s=2\\
  \end{cases}
\]
\note{Method 1} Let $X = \vec{S}_1 \cdot \vec{S}_2$, we construct a second-degree polynomial $p(X)$ with the following properties
\[
  p(X) = 
  \begin{cases}
    0 & X=-2\\
    0 & X=-1\\
    2 & X=2\\
  \end{cases}
\]
This can be constructed as
\[
  p(X) = \frac{1}{6}(X+2)(X+1) = \frac{X^2}{6} + \frac{X}{2} + \frac{1}{3}
\]
which is the desired expression.
\note{Method 2} Alternatively, we construct a projector in terms of the total angular momentum that gives $\Pi_{s=2}=0$ for $s=0,1$ and $\Pi_{s=2}=1$ for $s=2$.
\[
  \Pi_{s=2} = \frac{1}{24}\underbrace{\left( (\vec{S}_1 + \vec{S}_2)^2 -2\right)}_{=0~\text{for}~\ket{s=1}}\underbrace{\left(\vec{S}_1 + \vec{S}_2)^2\right)}_{=0~\text{for}~\ket{s=0}} = \frac{1}{2} \vec{S}_1 \cdot \vec{S}_2 + \frac{1}{6}  (\vec{S}_1 \cdot \vec{S}_2)^2 + \frac{1}{3} = h_{1,2} .
\]
\end{section}
\end{document}
%%%%%%%%%%% Here Ends Document %%%%%%%%%%
